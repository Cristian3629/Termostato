\documentclass[10pt,a4paper]{article}
\usepackage[latin1]{inputenc}
\usepackage[spanish]{babel}
\usepackage{amsmath}
\usepackage{amsfonts}
\usepackage{amssymb}
\usepackage{graphicx}
\usepackage{vmargin}
\setpapersize{A4}
\setmargins
{2.5cm}       % margen izquierdo
{1.5cm} % margen superior
{16.5cm}% anchura del texto
{23.42cm} % altura del texto
{10pt} % altura de los encabezados
{1cm} % espacio entre el texto y los encabezados
{0pt} % altura del pie de p�gina
{2cm}% espacio entre el texto y el pie de p�gina

\begin{document}
	
	\begin{titlepage}
		\centering
		{\scshape\LARGE Universidad de Buenos Aires \par}
		\vspace{1cm}
		{\scshape\Large Taller de Programaci�n (75.42)\par}
		\vspace{1.5cm}
		{\huge\bfseries Termostato Inteligente \par}
		\vspace{2cm}
		{\Large\itshape Cristian Gonz�lez\par}
		\vspace{2cm}
		{\Large\itshape 94719 \par}
	\end{titlepage}
	
	
	
	
	
	
	%�ndice.
	\tableofcontents
	\newpage
\section{Archivos}
\subsection{File.c}
Este archivo se encarga de encapsular todo lo referente
al procesamiento del archivo tanto como abrirlo, leerlo,
procesarlo y liberar sus recursos correspondiente.

\subsection{Time.c}
Se encarga de la actualizaci�n constante, de la hora ingresada
por el usuario, cada vez que se envia todas las mediciones
recibidas por el Termostato.

\subsection{CalculateTemp.c}

Se centra en todo el proceso desde "transformar" el n�mero
hexadecimal recibido hasta llegar al n�mero decimal
que representa la temperatura.

\subsection{Principal.c}

Es el punto de entrada al programa, dado los p�rametros
ingresados, este decide si el programa se ejecutar� en
modo cliente o en modo servidor.

\subsection{Cliente.c}
Este es el modulo que modela todo el comportamiento del cliente, envia
la fecha como la hora y las mediciones capturadas.

\subsection{Lista.c y Nodo.c}
Estas 2 estructuras se utilizan para insertar los datos de forma ordenada para poder obtener la mediana.

\section{Aclaraciones}
\begin{enumerate}
	\item El cliente al leer un temperatura lo env�a y inmediatamente despu�s env�a un espacio o un '/n' dependiendo de la situaci�n.
\end{enumerate}

Link del repositorio:https://github.com/Cristian3629/Termostato

\end{document}